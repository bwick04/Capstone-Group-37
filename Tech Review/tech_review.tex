\documentclass[letterpaper,10pt,onecolumn,draftclsnofoot]{IEEEtran}
\usepackage{times}

\usepackage[english]{babel}
\usepackage[margin=0.75in]{geometry}

\usepackage{graphicx}

\DeclareGraphicsExtensions{.pdf,.png,.jpg}

\title{Object Speed Tracking}
\author{Tech Review\\Alex Bailey, Ben Wick, Dylan Washburne\\CS 461, Fall Term}

\begin{document}

\begin{titlepage}

\maketitle

\begin{abstract}
Using a stationary camera, with the intent of being mounted on a car, we are attempting to  detect objects and determine the speeds of those objects relative to the Observer (camera).
This will be done by having the camera recognize objects in space and determine their speeds based on the rate at which they travel through the frame.
If the camera is on a moving object, then we will need to either have a way for the system to measure it's own speed, likely with an accelerometer or connect to the object, if the object is measuring it's own speed.
To make this work, we will have to research the varieties of cameras available to use, as well as the API’s  they operate with.
We will also have to review the available computer vision algorithms and determine which is the most appropriate.
From this, we will determine the best camera to be used and from there create a object tracking program.
 
\end{abstract}

\end{titlepage}

\section{Problem Definition}

%possible sections:  

%how the video feed is stored (1 feed+data, 2 feeds no data to be re-calculated, 1 feed data overlayed)
%connection speeds
%live compression
%long-term compression
%video to database storage
%mono- vs bi-focal
%computer vision
%speed formula
%legal accountability
%internal clocks/global shutters

\end{document}
