\documentclass[letterpaper,10pt,onecolumn,draftclsnofoot]{IEEEtran}
\usepackage{times}

\usepackage[english]{babel}
\usepackage[margin=0.75in]{geometry}

\usepackage{graphicx}

\DeclareGraphicsExtensions{.pdf,.png,.jpg}

\title{Object Speed Tracking}
\author{Tech Review\\Alex Bailey, Ben Wick, Dylan Washburne\\CS 461, Fall Term}

\begin{document}

\begin{titlepage}

\maketitle

\begin{abstract}
While we could consider whether we would use a singular camera or multiple, there is really no debate here.
Stereoscopic cameras carry such an advantage for a project like this, we are making that decision.
 
\end{abstract}

\end{titlepage}

\section{Live Data Feed} %Alex

Both images simultaneously
Both images plus depth information
Singular image with depth

\section{Live Compression} %Dylan

No compression
Camera live compresses
Computer live compresses

\section{Long-Term Storage}%Alex

1 feed+data, 
2 feeds no data to be re-calculated, 
1 feed data baked into video

\section{CV Software Package} %Ben

OpenCV
vlfeat
vxl

\section{CV Underlying Algorithm} %Ben

Harr cascades
...
...

\section{Synchronization} %Alex

two cameras, each sends images and computer works to compensate for any desyncronization
two cameras with global shutter, send back 2 images
2 cameras which do image splicing and send back spliced image to computer

\section{UI Overlay} %Dylan

Open Broadcaster Software
VideoMeld
Camtasia

Goals for the selected software are to show the video stream as it is given, as well as placing boses around objects points sent from the backend's recognition software.

Criteria for each of these softwares include performance overhead and the ability to add more overlay components in real time.

\begin{tabular}{ l l }
  OBS Studio & OBS is a fairly common overlay software.  It has a stable level of performance and is able to pop in elements for the overlay at any time.  It is undetermined as of yet if this software would be able to make these additional elements go to dynamic positions on screen. \\ \hline
  VideoMeld & This seems to be from before the era of livestreams, and though it has a wide variety for overlay objects, I cannot find any way for it to support dynamically occuring objects in the overlay  \\ \hline
  Camtasia & This only works live through an extra plugin and it has significant overhead in the process.  Dynamic overlay objects do not seem to be supported. \\
\end{tabular}

Discussion

I choose OBS based on what is discussed above.


\section{Existing Speed Formulas} %Dylan

make our own
...
...

\section{Long-Term Compression} %Alex 
%Lossless
%Lossy
%No Compression

The goals for this piece of the project is to compress the video after it has been displayed to the user, to be kept for long term, should they be needed at a later date.
With our product, it is likely that the user will be leaving our product running for a significant amount of time, possibly hours.
When this happens, video files can become rather large.
This will become a problem for long term storage if our product is used often.
So, the answer to this problem is to use a compression codec.
This will reduce the size of the video as much as possible.
There are several factors to consider when looking at video compression codecs.
The first is whether or not it is lossless.
When compressing information, especially pictures and videos, the compression codecs will often save space by removing data, often in the form of merging pixels, leading to a lower resolution.
Many people are willing to accept the lower resolution.
But, because our customers may need to reuse the video to recalculate the velocities of the objects at a later date, we will need a codec that is lossless.
Second is the amount of compression, often expressed as a ratio.
This is the ration of the size of the original video file to the size of the compressed file.
While this
Third is the speed of the compression.



%possible sections:  

%how the video feed is stored (1 feed+data, 2 feeds no data to be re-calculated, 1 feed data overlayed)
%connection speeds
%live compression
%long-term compression
%video to database storage
%mono- vs bi-focal
%computer vision
%speed formula
%legal accountability
%internal clocks/global shutters (syncronization)

\end{document}
