\documentclass[letterpaper,10pt,onecolumn,draftclsnofoot]{IEEEtran}
\usepackage{times}

\usepackage[english]{babel}
\usepackage[margin=0.75in]{geometry}

\usepackage{graphicx}

\DeclareGraphicsExtensions{.pdf,.png,.jpg}

\title{Object Speed Tracking}
\author{Tech Review\\Alex Bailey, Ben Wick, Dylan Washburne\\CS 461, Fall Term}

\begin{document}

\begin{titlepage}

\maketitle

\begin{abstract}
While we could consider whether we would use a singular camera or multiple, there is really no debate here.
Stereoscopic cameras carry such an advantage for a project like this, we are making that decision.
 
\end{abstract}

\end{titlepage}

\section{Live Data Feed} %Alex

Both images simultaneously
Both images plus depth information
Singular image with depth

\section{Live Compression} %Dylan

No compression
Camera live compresses
Computer live compresses

\section{Long-Term Storage}%Alex

1 feed+data, 
2 feeds no data to be re-calculated, 
1 feed data baked into video

\section{Computer Vision Library} %Ben

The three options for Computer Vision software packages include OpenCV, VXL, and AForge.
Selecting a good Computer Vision library is essential. The goal of the CV library is to provide us with a large array of programming functions that we are able to utilize.
The library selected must be able to support our needs of being able to identify and track objects in real time.
The criteria that will be evaluated are languages used, features available, and performance.
OpenCV is one of the most commonly used libraries for computer vision.
According to their website, they offer over 2500 algorithms that include identifying objects.
AForge is also another Computer vision library that is able to detect the motion of objects.
AForge isn't as commonly used as OpenCV but it is also a great alternative because it also offers motion tracking.
Based on the criteria needed for computer vision library, OpenCV has a large number of algorithms that we will be able to use as well as performs faster than VXL.

\section{Computer Vision Underlying Algorithm} %Ben

Haar cascades
Background subraction

...
...

\section{Synchronization} %Alex

two cameras, each sends images and computer works to compensate for any desyncronization
two cameras with global shutter, send back 2 images
2 cameras which do image splicing and send back spliced image to computer

\section{UI Overlay} %Dylan

Open Broadcastcast Software
...
...

\section{Existing Speed Formulas} %Dylan

make our own
...
...

\section{Long-Term Compression} %Alex 

Lossless
Lossy
No Compression


%possible sections:  

%how the video feed is stored (1 feed+data, 2 feeds no data to be re-calculated, 1 feed data overlayed)
%connection speeds
%live compression
%long-term compression
%video to database storage
%mono- vs bi-focal
%computer vision
%speed formula
%legal accountability
%internal clocks/global shutters (syncronization)

\end{document}
